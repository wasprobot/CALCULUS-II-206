\documentclass{article}
\usepackage{amsmath,amssymb}
\newcounter{question}
\setcounter{question}{0}
\begin{document}

\newcommand\Que[1]{%
   \leavevmode\par
   \stepcounter{question}
   \noindent
   \thequestion. Q --- #1\par}

\newcommand\Ans[2][]{%
    \leavevmode\par\noindent
   {A --- \textbf{#1}#2\par}}

\Que{ $ \int{e^{8\theta}\sin(9\theta)}d\theta $ }
\Ans
{
Let $ u=e^{8\theta} $ and $ v^{\prime}=\sin(9\theta) $.\\

$\implies u^{\prime} = 8e^{8\theta} $ and $ v = -\frac{\cos(9\theta)}{9} $\\\\

According to integration by parts:\\
$$\int{uv^{\prime}}=uv-\int{vu^{\prime}}$$\\

Therefore $ \int{e^{8\theta}\sin(9\theta)}d\theta $\\

$ = \frac{-e^{8\theta}\cos(9\theta)}{9} + \frac{8}{9}\int{e^{8\theta}\cos(9\theta)} d\theta $\\\\

To calculate $ \int{e^{8\theta}\cos(9\theta)} d\theta $\\

Let $ u=e^{8\theta} $ and $ v^{\prime}=\cos(9\theta) $.\\

$\implies u^{\prime} = 8e^{8\theta} $ and $ v = \frac{\sin(9\theta)}{9} $\\\\

Therefore $ \int{e^{8\theta}\cos(9\theta)} d\theta $\\

$ = \frac{e^{8\theta}\sin(9\theta)}{9} - \frac{8}{9}\int{e^{8\theta}\sin(9\theta)} d\theta $\\\\

Therefore $ \int{e^{8\theta}\sin(9\theta)}d\theta $\\

$ = \frac{-e^{8\theta}\cos(9\theta)}{9} + \frac{8}{9}[\frac{e^{8\theta}\sin(9\theta)}{9} - \frac{8}{9}\int{e^{8\theta}\sin(9\theta)} d\theta] $\\\\

$ = \frac{-e^{8\theta}\cos(9\theta)}{9} + \frac{8}{81}e^{8\theta}\sin(9\theta) - \frac{64}{81}\int{e^{8\theta}\sin(9\theta)} d\theta $\\\\

$ (1+\frac{64}{81})\int{e^{8\theta}\sin(9\theta)} d\theta = \frac{-e^{8\theta}\cos(9\theta)}{9} + \frac{8}{81}e^{8\theta}\sin(9\theta) $\\\\

$ \frac{145}{81}\int{e^{8\theta}\sin(9\theta)} d\theta = \frac{-9}{81}e^{8\theta}\cos(9\theta) + \frac{8}{81}e^{8\theta}\sin(9\theta) $\\\\

$ \int{e^{8\theta}\sin(9\theta)} d\theta = \frac{-9}{145}e^{8\theta}\cos(9\theta) + \frac{8}{145}e^{8\theta}\sin(9\theta) + C $\\

}
\end{document}