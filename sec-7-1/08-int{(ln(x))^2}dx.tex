\documentclass{article}
\usepackage{amsmath,amssymb}
\newcounter{question}
\setcounter{question}{0}
\begin{document}

\newcommand\Que[1]{%
   \leavevmode\par
   \stepcounter{question}
   \noindent
   \thequestion. Q --- #1\par}

\newcommand\Ans[2][]{%
    \leavevmode\par\noindent
   {A --- \textbf{#1}#2\par}}

\Que{ $ \int{(\ln(x))^2}dx $ }
\Ans
{
Let $ u=(\ln(x))^2 $ and $ v^{\prime}=1 $.\\

$\frac{d}{dx}(\ln(x))^2 = 2\ln(x) \frac{d}{dx}\ln(x) $\\

$ = \frac{2\ln(x)}{x} $\\

Therefore $ u^{\prime}=\frac{2\ln(x)}{x} $ and $ v=x $\\

According to integration by parts:\\
$$\int{uv^{\prime}}=uv-\int{vu^{\prime}}$$\\

Therefore $ \int{(\ln(x))^2}dx = x(\ln(x))^2 - 2\int{\ln(x)} dx $\\

To calculate $ \int{\ln(x)} dx $, let $ u=\ln(x) $ and $ v^{\prime}=1 $.\\

Therefore $ u^{\prime}=\frac{1}{x} $ and $ v=x $\\

Therefore $ \int{\ln(x)} dx = x\ln(x) - \int{1} dx$\\

$ = x\ln(x) - x + C_1 $\\

Therefore $ \int{(\ln(x))^2}dx = x(\ln(x))^2 - 2[x\ln(x) - x] + C $\\

$ = x(\ln(x))^2 - 2x\ln(x) + 2x + C $\\

}
\end{document}