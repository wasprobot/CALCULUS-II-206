\documentclass{article}
\usepackage{amsmath,amssymb}
\newcounter{question}
\setcounter{question}{0}
\begin{document}

\newcommand\Que[1]{%
   \leavevmode\par
   \stepcounter{question}
   \noindent
   \thequestion. Q --- #1\par}

\newcommand\Ans[2][]{%
    \leavevmode\par\noindent
   {A --- \textbf{#1}#2\par}}

\Que{ $ \int_{0}^{1}{(x^2+9)e^{-x}} dx $ }
\Ans
{
Let $ u = x^2+9 $ and $ v^{\prime} = e^{-x} 
\implies u^{\prime} = 2x $ and $ v = -e^{-x} $\\\\

According to integration by parts:\\
$$\int{uv^{\prime}}=uv-\int{vu^{\prime}}$$\\

Therefore $ \int{(x^2+9)e^{-x}} dx 
= -(x^2+9)e^{-x} + 2 \int{ xe^{-x} } dx $\\\\

To calculate $ \int{ xe^{-x} } dx $\\

Let $ u = x $ and $ v^{\prime} = e^{-x} 
\implies u^{\prime} = 1 $ and $ v = -e^{-x} $\\

Therefore $ \int{ xe^{-x} } dx 
= -xe^{-x} + \int{e^{-x}} dx
= -xe^{-x} - e^{-x}
= -(x+1)e^{-x} $\\\\

Therefore $ \int{(x^2+9)e^{-x}} dx 
= -(x^2+9)e^{-x} + 2[-(x+1)e^{-x}] $\\

$ = -(x^2+2x+11)e^{-x} $\\\\

Therefore $ \int_{0}^{1}{(x^2+9)e^{-x}} dx 
= \left[ -(x^2+2x+11)e^{-x} \right]_{0}^{1} $\\

$ = -\left[ (x^2+2x+11)e^{-x} \right]_{0}^{1} $\\

$ = -(14e^{-1} - 11e^{-0}) $\\

$ = 11 - \frac{14}{e} $\\

}
\end{document}