\documentclass{article}
\usepackage{amsmath,amssymb}
\newcounter{question}
\setcounter{question}{0}
\begin{document}

\newcommand\Que[1]{%
   \leavevmode\par
   \stepcounter{question}
   \noindent
   \thequestion. Q) #1\par}

\newcommand\Ans[2][]{%
    \leavevmode\par\noindent
   {A) \textbf{#1}#2\par}}

\Que{
    Find the exact length of the curve
    $ f(x) = \frac{x^2}{2}, P = (-3, \frac{9}{2}), Q = (3, \frac{9}{2}) $\\
    }
\Ans{
    $ L = \int_{-3}^{3} {\sqrt{1+[f^{\prime}(x)]^2}} dx $\\
    
    $ f^{\prime}(x) = \frac{2x}{2} = x $\\

    $ L = \int_{-3}^{3} {\sqrt{1+x^2}} dx $\\
    
    $
    \int{\sqrt{1+x^2}} dx
    = \int{\sqrt{1+(\tan{\theta})^2}} (\sec{\theta})^2 d\theta
    $\\

    for $
    x = \tan{\theta},
    dx = (\sec{\theta})^2 d\theta
    $\\

    $
    \int{\sqrt{1+x^2}} dx
    = \int{\sqrt{1+(\tan{\theta})^2}} (\sec{\theta})^2 d\theta
    = \int{(sec{\theta})^3} d\theta
    $\\

    $
    = \int{\frac{1}{(\cos{\theta})^2} \frac{1}{\cos{\theta}} } d\theta
    $\\

    (applying integration by parts $ \int{u.dv} = u.v - \int{v.du} $),\\

    $
    = \sec{\theta}\tan{\theta} 
    - \int{\sec{\theta}(\tan{\theta})^2} d\theta
    $\\
    
    (for $
    u = \frac{1}{\cos{\theta}},
    dv = \frac{1}{(\cos{\theta})^2}
    $
    and $
    du = \sec{\theta}\tan{\theta}d\theta,
    v = \tan{\theta}
    $)\\

    $
    \implies \int{(\sec{\theta})^3}d{\theta}
    = \sec{\theta}\tan{\theta}
    - \int{(\sec{\theta})^3(\sin{\theta})^2} d\theta
    $\\

    $
    = \sec{\theta}\tan{\theta}
    - \int{(\sec{\theta})^3(1 - \cos{\theta})^2} d\theta
    $\\

    $
    = \sec{\theta}\tan{\theta}
    - \int{(\sec{\theta})^3} d{\theta} 
    + \int{\sec{\theta}} d\theta
    $\\

    $
    = \sec{\theta}\tan{\theta}
    - \int{(\sec{\theta})^3} d{\theta} 
    + \ln|\sec{\theta} + \tan{\theta}|
    $\\

    $
    \implies 2\int{(\sec{\theta})^3}d{\theta}
    = \sec{\theta}\tan{\theta}
    + \ln|\sec{\theta} + \tan{\theta}|
    $\\
    
    $
    \implies \int{(\sec{\theta})^3}d{\theta}
    = \frac{1}{2}[
        \sec{\theta}\tan{\theta}
        + \ln|\sec{\theta} + \tan{\theta}|
        ]
    $\\
    
    
    $
    = \frac{1}{2}[
        \sec{\theta}\tan{\theta}
        + \ln|\sec{\theta} + \tan{\theta}|
        ]
        $\\

    $ \implies L 
    = \frac{1}{2} \left[
        \sec{\theta}\tan{\theta}
        + \ln|\sec{\theta} + \tan{\theta}|
    \right]_{x=-3}^{x=3}
    $\\

    For $ x=\pm{3} $, $\tan{\theta}=\pm{3}, \sec{\theta}=\sqrt{10}$\\
    
    $ \implies L 
    = \frac{1}{2} \left[
        3\sqrt{10} + \ln|\sqrt{10} + 3|
        -(-3\sqrt{10} + \ln|\sqrt{10} - 3|)
    \right]$\\

    $
    = 3\sqrt{10} +  \frac{1}{2}\ln(\frac{\sqrt{10} + 3}{\sqrt{10} - 3})
    $\\

    }
\end{document}